% !TEX root = main.tex

\documentclass{exam}
\usepackage{graphicx} % Required for inserting images
\usepackage[french]{babel}
\usepackage{listings}
\usepackage{xcolor}
\usepackage{spreadtab}
\usepackage{amsmath}
\usepackage[T1]{fontenc}
\usepackage{hyperref}
\usepackage{pgf-pie}
\usepackage{bohr}
\usetikzlibrary{optics}
\usetikzlibrary{shapes.misc}
\usetikzlibrary{calc}
\usetikzlibrary{backgrounds}
\tikzset{cross/.style={cross out, draw=black, fill=none, minimum size=2*(#1-\pgflinewidth), inner sep=0pt, outer sep=0pt}, cross/.default={10pt}}
\hypersetup{
    colorlinks=true,
    linkcolor=blue,
    filecolor=magenta,      
    urlcolor=cyan,
    pdftitle={Atelier Lisbonne},
    pdfpagemode=FullScreen,
    }
\usepackage[margin=2cm]{geometry}
\usepackage{media9}
\usepackage{tabularx}
\usepackage{biocon}
\usepackage{pgf-spectra}
\usepackage{verbatim}
\definecolor{codegreen}{rgb}{0,0.6,0}
\definecolor{codegray}{rgb}{0.5,0.5,0.5}
\definecolor{codepurple}{rgb}{0.58,0,0.82}
\definecolor{backcolour}{rgb}{0.95,0.95,0.92}

%Code listing style named "mystyle"
\lstdefinestyle{mystyle}{
  backgroundcolor=\color{backcolour}, commentstyle=\color{codegreen},
  keywordstyle=\color{magenta},
  numberstyle=\tiny\color{codegray},
  stringstyle=\color{codepurple},
  basicstyle=\ttfamily\footnotesize,
  breakatwhitespace=false,         
  breaklines=true,                 
  captionpos=b,                    
  keepspaces=true,                 
  numbers=left,                    
  numbersep=5pt,                  
  showspaces=false,                
  showstringspaces=false,
  showtabs=false,                  
  tabsize=2
}

%"mystyle" code listing set
\lstset{style=mystyle}
\newcommand{\soll}[1]{%
\ifprintanswers
        #1%
\fi
}

% Définition de la macro TikZ
\newcommand{\drawOpticsLens}[3]{%
    % Paramètres : #1 = tailleObjet, #2 = distanceImage, #3 = focale
    \begin{tikzpicture}[use optics]

    % === Paramètres donnés ===
 %   \pgfmathsetmacro{\tailleObjet}{#1}
 %   \pgfmathsetmacro{\distanceImage}{#2}
 %   \pgfmathsetmacro{\focale}{#3}

    % Calcul automatique des autres valeurs
  %  \pgfmathsetmacro{\tailleImage}{\distanceImage*\tailleObjet/\focale}
  %  \pgfmathsetmacro{\distanceObjet}{\focale*\tailleObjet/\tailleImage}
 %   \pgfmathsetmacro{\hauteurIntersection}{\focale*\tailleObjet/\distanceObjet}

    \pgfmathsetmacro{\tailleObjet}{#1}
    \pgfmathsetmacro{\focale}{#2}
    \pgfmathsetmacro{\distanceObjet}{#3}
    
    \pgfmathsetmacro{\tailleImage}{\focale*\tailleObjet/\distanceObjet}
    \pgfmathsetmacro{\distanceImage}{\tailleImage*\focale/\tailleObjet}
    \pgfmathsetmacro{\hauteurIntersection}{\focale*\tailleObjet/\distanceObjet}
    \pgfmathsetmacro{\tailleLentille}{1.2*(\tailleObjet+\tailleImage)}
    
    % === Lentille ===
    \node[lens, draw,
          draw focal points={minimum width=5pt,minimum height=5pt},
          focal length=\focale cm,
          object height=\fpeval{\tailleLentille}cm
          ] (L) at (0,0) {};

    % === Objet ===
    \coordinate (A) at ($(L.west focus)+(-\distanceObjet cm,0)$);
    \coordinate (B) at ($(L.west focus)+(-\distanceObjet cm,+\tailleObjet)$);
    \node[below=0.1cm] at (A) {A};
    \node[left=0.1cm] at (B) {B};
    \draw[->,thick,blue] (A) -- (B);

    % === Image ===
    \coordinate (AA) at ($(L.east focus)+(\distanceImage cm,0)$);
    \coordinate (BB) at ($(L.east focus)+(\distanceImage cm,-\tailleImage cm)$);
    \node[above=0.1cm] at (AA) {A'};
    \node[right=0.1cm] at (BB) {B'};
    \draw[->,thick,blue] (AA) -- (BB);

    % === Foyers et axe optique ===
    \node[below=0.2cm] at (L.west focus) {F};
    \node[below=0.2cm] at (L.east focus) {F'};
    \node[above] at ($(A)+(1.2cm,0)$) {axe optique $\Delta$};
    \draw[dashed, thick] ($(A)+(-0.5cm,0)$) -- ($(AA)+(0.5cm,0)$);
    \node[below=0.2cm, left=0] at (0,0) {O};

    % === Rayons principaux ===
    \draw[blue] (B) -- (0,\tailleObjet) -- (BB);
    \draw[red] (B) -- (BB);
    \draw[green] (B) -- (0,-\hauteurIntersection) -- (BB);
        
    % === Grille dynamique ===
    \pgfmathsetmacro{\margin}{1} % marge autour de l'objet et de l'image
    \begin{pgfonlayer}{background}
    \draw[step=1.0, yellow, thin] 
        ($(B)+(-\margin,\margin)$) grid ($(BB)+(\margin,-\margin)$);
    \end{pgfonlayer}

    

    \end{tikzpicture}
}

\newcommand{\red}{\.[{.style={draw=red,fill=red}}]} % red dot for single bond in draft mode
\newcommand{\redd}{\:[{.style={draw=red,fill=red}}]}% red dots for double bond in draft mode
\newcommand{\draftlp}{\:}   % draft mode for lone pair
\usepackage{forest}
\usepackage{siunitx}
\sisetup{locale=FR,output-decimal-marker={,}}
\usepackage{elements}
\usepackage{chemfig}
\usepackage{upgreek}
\usepackage{chemgreek}
\usepackage{chemmacros}


\unframedsolutions
\renewcommand{\solutiontitle}{}
\SolutionEmphasis{\color{red}}
