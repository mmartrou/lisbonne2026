\documentclass{article}
\usepackage{graphicx} % Required for inserting images
\usepackage[french]{babel}
\usepackage{listings}
\usepackage{xcolor}
\usepackage{spreadtab}
\usepackage{amsmath}
\definecolor{codegreen}{rgb}{0,0.6,0}
\definecolor{codegray}{rgb}{0.5,0.5,0.5}
\definecolor{codepurple}{rgb}{0.58,0,0.82}
\definecolor{backcolour}{rgb}{0.95,0.95,0.92}

%Code listing style named "mystyle"
\lstdefinestyle{mystyle}{
  backgroundcolor=\color{backcolour}, commentstyle=\color{codegreen},
  keywordstyle=\color{magenta},
  numberstyle=\tiny\color{codegray},
  stringstyle=\color{codepurple},
  basicstyle=\ttfamily\footnotesize,
  breakatwhitespace=false,         
  breaklines=true,                 
  captionpos=b,                    
  keepspaces=true,                 
  numbers=left,                    
  numbersep=5pt,                  
  showspaces=false,                
  showstringspaces=false,
  showtabs=false,                  
  tabsize=2
}

%"mystyle" code listing set
\lstset{style=mystyle}

\title{Atelier à Lisbonne sur \LaTeX}
\author{Guillaume Martrou}
\date{\today}

\begin{document}

\maketitle

\tableofcontents

\section{Introduction à \LaTeX}

\subsection{Des exemples}

Texte en \textit{italique}.
Texte en \textbf{gras}.
Texte en \textbf{\textit{gras et italique}}.
Texte en emphase en fonction du contexte. Ici \emph{ce mot}.
\textbf{Dans cette phrase, \emph{ce mot}}.
\textit{Dans cette phrase, \emph{ce mot}}.

\begin{lstlisting}[language=Python, caption=Exemple python]
def sommeEntiers(n):
    for i in range(n+1):
        resultat += i
return resultat
\end{lstlisting}

\section{Une section numérotée}
\subsection{Une sous section numérotée}
\subsection*{Une sous section non numérotée}
\section*{Une section non numérotée}
\section{Encore une section}

\subsection{Listes}

\begin{itemize}
    \item Premier élément
    \item Deuxième élément 
    \begin{itemize}
        \item Sous élément
        \item sous élément
    \end{itemize}
    \item Autre
\end{itemize}

\begin{enumerate}
   \item Premier élément
    \item Deuxième élément 
    \begin{enumerate}
        \item Sous élément
        \begin{enumerate}
            \item Sous élément
            \item sous élément
        \end{enumerate}
        \item sous élément
    \end{enumerate}
    \item Autre 
\end{enumerate}

\subsection{Feuilles de calculs}

\begin{spreadtab}{{tabular}{ccccc}}
1 & & & & \\
a1 & a1 & & & \\
a2 & a2+b2 & b2 & & \\
a3 & a3+b3 & b3+c3 & c3 & \\
a2 & a4+b4 & b4+c4 & c4+d4 & d4
\end{spreadtab}

$
\begin{spreadtab}{{matrix}{}}
1\\
[0,-1] & [-1,-1]\\
[0,-1] & [-1,-1]+[0,-1] & [-1,-1]\\
[0,-1] & [-1,-1]+[0,-1] & [-1,-1]+[0,-1] & [-1,-1]\\
[0,-1] & [-1,-1]+[0,-1] & [-1,-1]+[0,-1] & [-1,-1]+[0,-1] & [-1,-1]
\end{spreadtab}
$

Table de multiplication utilisant une feuille de calcul :

\begin{spreadtab}{{tabular}{|c|*{10}{c}|}}
\hline
@$\times$           & 1                     & \STcopy{>}{b1+1}  & & & & & & & & \\
\hline
1                   & \STcopy{>,v}{!a2*b!1} &                   & & & & & & & & \\
\STcopy{v}{a2+1}    &                       &                   & & & & & & & & \\
                    &                       &                   & & & & & & & & \\
                    &                       &                   & & & & & & & & \\
                    &                       &                   & & & & & & & & \\
                    &                       &                   & & & & & & & & \\
                    &                       &                   & & & & & & & & \\
                    &                       &                   & & & & & & & & \\
                    &                       &                   & & & & & & & & \\
                    &                       &                   & & & & & & & & \\
\hline
\end{spreadtab}

\end{document}
