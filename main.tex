% !TEX root = main.tex

\documentclass{exam}
\usepackage{graphicx} % Required for inserting images
\usepackage[french]{babel}
\usepackage{listings}
\usepackage{xcolor}
\usepackage{spreadtab}
\usepackage{amsmath}
\usepackage[T1]{fontenc}
\usepackage{hyperref}
\usepackage{pgf-pie}
\usepackage{bohr}
\usetikzlibrary{optics}
\usetikzlibrary{shapes.misc}
\usetikzlibrary{calc}
\usetikzlibrary{backgrounds}
\tikzset{cross/.style={cross out, draw=black, fill=none, minimum size=2*(#1-\pgflinewidth), inner sep=0pt, outer sep=0pt}, cross/.default={10pt}}
\hypersetup{
    colorlinks=true,
    linkcolor=blue,
    filecolor=magenta,      
    urlcolor=cyan,
    pdftitle={Atelier Lisbonne},
    pdfpagemode=FullScreen,
    }
\usepackage[margin=2cm]{geometry}
\usepackage{media9}
\usepackage{tabularx}
\usepackage{biocon}
\usepackage{pgf-spectra}
\usepackage{verbatim}
\definecolor{codegreen}{rgb}{0,0.6,0}
\definecolor{codegray}{rgb}{0.5,0.5,0.5}
\definecolor{codepurple}{rgb}{0.58,0,0.82}
\definecolor{backcolour}{rgb}{0.95,0.95,0.92}

%Code listing style named "mystyle"
\lstdefinestyle{mystyle}{
  backgroundcolor=\color{backcolour}, commentstyle=\color{codegreen},
  keywordstyle=\color{magenta},
  numberstyle=\tiny\color{codegray},
  stringstyle=\color{codepurple},
  basicstyle=\ttfamily\footnotesize,
  breakatwhitespace=false,         
  breaklines=true,                 
  captionpos=b,                    
  keepspaces=true,                 
  numbers=left,                    
  numbersep=5pt,                  
  showspaces=false,                
  showstringspaces=false,
  showtabs=false,                  
  tabsize=2
}

%"mystyle" code listing set
\lstset{style=mystyle}
\newcommand{\soll}[1]{%
\ifprintanswers
        #1%
\fi
}

% Définition de la macro TikZ
\newcommand{\drawOpticsLens}[3]{%
    % Paramètres : #1 = tailleObjet, #2 = distanceImage, #3 = focale
    \begin{tikzpicture}[use optics]

    % === Paramètres donnés ===
 %   \pgfmathsetmacro{\tailleObjet}{#1}
 %   \pgfmathsetmacro{\distanceImage}{#2}
 %   \pgfmathsetmacro{\focale}{#3}

    % Calcul automatique des autres valeurs
  %  \pgfmathsetmacro{\tailleImage}{\distanceImage*\tailleObjet/\focale}
  %  \pgfmathsetmacro{\distanceObjet}{\focale*\tailleObjet/\tailleImage}
 %   \pgfmathsetmacro{\hauteurIntersection}{\focale*\tailleObjet/\distanceObjet}

    \pgfmathsetmacro{\tailleObjet}{#1}
    \pgfmathsetmacro{\focale}{#2}
    \pgfmathsetmacro{\distanceObjet}{#3}
    
    \pgfmathsetmacro{\tailleImage}{\focale*\tailleObjet/\distanceObjet}
    \pgfmathsetmacro{\distanceImage}{\tailleImage*\focale/\tailleObjet}
    \pgfmathsetmacro{\hauteurIntersection}{\focale*\tailleObjet/\distanceObjet}
    \pgfmathsetmacro{\tailleLentille}{1.2*(\tailleObjet+\tailleImage)}
    
    % === Lentille ===
    \node[lens, draw,
          draw focal points={minimum width=5pt,minimum height=5pt},
          focal length=\focale cm,
          object height=\fpeval{\tailleLentille}cm
          ] (L) at (0,0) {};

    % === Objet ===
    \coordinate (A) at ($(L.west focus)+(-\distanceObjet cm,0)$);
    \coordinate (B) at ($(L.west focus)+(-\distanceObjet cm,+\tailleObjet)$);
    \node[below=0.1cm] at (A) {A};
    \node[left=0.1cm] at (B) {B};
    \draw[->,thick,blue] (A) -- (B);

    % === Image ===
    \coordinate (AA) at ($(L.east focus)+(\distanceImage cm,0)$);
    \coordinate (BB) at ($(L.east focus)+(\distanceImage cm,-\tailleImage cm)$);
    \node[above=0.1cm] at (AA) {A'};
    \node[right=0.1cm] at (BB) {B'};
    \draw[->,thick,blue] (AA) -- (BB);

    % === Foyers et axe optique ===
    \node[below=0.2cm] at (L.west focus) {F};
    \node[below=0.2cm] at (L.east focus) {F'};
    \node[above] at ($(A)+(1.2cm,0)$) {axe optique $\Delta$};
    \draw[dashed, thick] ($(A)+(-0.5cm,0)$) -- ($(AA)+(0.5cm,0)$);
    \node[below=0.2cm, left=0] at (0,0) {O};

    % === Rayons principaux ===
    \draw[blue] (B) -- (0,\tailleObjet) -- (BB);
    \draw[red] (B) -- (BB);
    \draw[green] (B) -- (0,-\hauteurIntersection) -- (BB);
        
    % === Grille dynamique ===
    \pgfmathsetmacro{\margin}{1} % marge autour de l'objet et de l'image
    \begin{pgfonlayer}{background}
    \draw[step=1.0, yellow, thin] 
        ($(B)+(-\margin,\margin)$) grid ($(BB)+(\margin,-\margin)$);
    \end{pgfonlayer}

    

    \end{tikzpicture}
}

\newcommand{\red}{\.[{.style={draw=red,fill=red}}]} % red dot for single bond in draft mode
\newcommand{\redd}{\:[{.style={draw=red,fill=red}}]}% red dots for double bond in draft mode
\newcommand{\draftlp}{\:}   % draft mode for lone pair
\usepackage{forest}
\usepackage{siunitx}
\sisetup{locale=FR,output-decimal-marker={,}}
\usepackage{elements}
\usepackage{chemfig}
\usepackage{upgreek}
\usepackage{chemgreek}
\usepackage{chemmacros}


\unframedsolutions
\renewcommand{\solutiontitle}{}
\SolutionEmphasis{\color{red}}


\title{Atelier à Lisbonne sur \LaTeX}
\author{Guillaume Martrou\\guillaume.martrou@lfmadrid.org}
\date{\today}

\noprintanswers

\begin{document}

\maketitle

\tableofcontents

\section{Introduction}

\subsection{Mise en forme}
Voici la \textbf{première} \textit{phrase} de l'atelier sur \LaTeX. Les deux derniers mots de cette phrase sont \textcolor{red}{en rouge}.

Comme il s'agit d'un langage de programmation, on peut écrire le nombre \numberstringnum{4242424242} en toute lettre facilement. Lister beaucoup de valeurs ayant la même unité comme \qtylist{1;2;3;4;5;6;8}{\m\per\s\squared}. Et noter rapidement des puissances de 10 comme \num{3e578}.

\subsection{Des tableaux et des images}

\begin{figure}
    \centering
    \includegraphics[width=0.25\linewidth]{Photosynthese.png}
    \caption{Schéma de la photosynthèse}
    \label{fig:photosynthese}
\end{figure}
\begin{table}
    \centering
    \begin{tabular}{cc}
        Symbole de l'élément & numéro atomique (Z) \\
        H & 1 \\
        C & 6 \\
    \end{tabular}
    \caption{Numéro atomique d'éléments chimiques}
    \label{tab:numeroAtomique}
\end{table}

On peut    laisser     autant   d'espaces que l'on veut. Et On
    peut
    écrire
    une
    phrase
    sur
    plusieurs
    lignes.

\LaTeX{} essaye de :
\begin{enumerate}
    \item rendre le document le mieux agencé possible.
    \item placer les images et les tableaux au meilleur endroit.
\end{enumerate}

Et il permet de :

\begin{itemize}
    \item référencer les images, comme l'image \ref{fig:photosynthese}.
    \item et les tableaux, comme le tableau \ref{tab:numeroAtomique}.
\end{itemize}

\section{Écrire des mathématiques}

\subsection{En ligne}

On peut ajouter, en ligne, des équations comme $\int \cos(x) dx = \sin(x)$. 

\subsection{Maths en valeur !}

\[
E = \dfrac{mc^2}{1-\frac{v^2}{c^2}}
\]

\begin{align*}
    P   &= m \times g\\
        &= \qty{10}{\kg} \times \qty{9.81}{\N\per\kg}\\
        &= \qty{98.1}{\N}
\end{align*}

\section{Autres packages}

\subsection{tikz}
\drawOpticsLens{2}{3.0}{4.0}

\pgfplotstableread[col sep=comma]{mesures.csv}\mydata
\begin{figure}[ht]
    \centering
    \begin{tikzpicture}
    \begin{axis}[
        width=0.9\textwidth,
        height=0.6\textwidth,
        xlabel={Distance à la source lumineuse (cm)},
        ylabel={Taux de photosynthèse (mm$^3$ O$_2$/min)},
        title={Influence de la distance à la source lumineuse sur la photosynthèse},
        grid=major,
        xmin=0, xmax=100,
        ymin=0, ymax=7,
        xtick={0,20,40,60,80,100},
        ytick={0,1,2,3,4,5,6,7},
        tick label style={font=\small},
        label style={font=\small}
    ]
    
    % Graphique des points expérimentaux
    \addplot[
        only marks,
        mark=*,
        mark size=3pt,
        color=blue
    ] table[
        x=Distance,
        y=Photosynthese,
        col sep=comma
    ] {mesures.csv};
    \addlegendentry{Données expérimentales}
    
    \addplot[red, thick, smooth] 
    table[x=Distance, y=Photosynthese] {\mydata};
    \addlegendentry{Courbe lissée}
    
    \end{axis}
    \end{tikzpicture}
    \caption{Résultats expérimentaux}
    \label{fig:resultats}
\end{figure}

\section*{Tableau des mesures}

\begin{center}
\pgfplotstabletypeset[
    col sep=comma,
    string type,
    columns={Distance,Photosynthese},
    columns/Distance/.style={
        column name={Distance (cm)},
        dec sep align
    },
    columns/Photosynthese/.style={
        column name={Production O$_2$ (mm³/min)},
        dec sep align,
        precision=1,
        dec sep={,},
        fixed,          % ← Ajouter
        fixed zerofill, % ← Ajouter
    },
    every head row/.style={
        before row=\toprule,
        after row=\midrule
    },
    every last row/.style={
        after row=\bottomrule
    },
    every even row/.style={
        before row={\rowcolor{blue!5}}
    }
]{mesures.csv}
\end{center}


\subsection{chemfig, chemmacros et elements}

On peut noter facilement les différents isotopes du lithium avec \isotope{Li} ou \isotope{8,Li}, écrire une équation de réaction avec \ch{CH4 + 2 O2 -> CO2 + 2 H2O}. La configuration électronique du plomb : \elconf{Pb}. Ou le schéma d'une molécule comme le fluorure d'hydrogène :

\chemfig{H-F}

\chemfig{H-C(-[2]H)(-[6]H)-H}

\end{document}
