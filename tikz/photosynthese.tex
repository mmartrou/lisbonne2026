\pgfplotstableread[col sep=comma]{mesures.csv}\mydata
\begin{figure}[ht]
    \centering
    \begin{tikzpicture}
    \begin{axis}[
        width=0.9\textwidth,
        height=0.6\textwidth,
        xlabel={Distance à la source lumineuse (cm)},
        ylabel={Taux de photosynthèse (mm$^3$ O$_2$/min)},
        title={Influence de la distance à la source lumineuse sur la photosynthèse},
        grid=major,
        xmin=0, xmax=100,
        ymin=0, ymax=7,
        xtick={0,20,40,60,80,100},
        ytick={0,1,2,3,4,5,6,7},
        tick label style={font=\small},
        label style={font=\small}
    ]
    
    % Graphique des points expérimentaux
    \addplot[
        only marks,
        mark=*,
        mark size=3pt,
        color=blue
    ] table[
        x=Distance,
        y=Photosynthese,
        col sep=comma
    ] {mesures.csv};
    \addlegendentry{Données expérimentales}
    
    \addplot[red, thick, smooth] 
    table[x=Distance, y=Photosynthese] {\mydata};
    \addlegendentry{Courbe lissée}
    
    \end{axis}
    \end{tikzpicture}
    \caption{Résultats expérimentaux}
    \label{fig:resultats}
\end{figure}

\section*{Tableau des mesures}

\begin{center}
\pgfplotstabletypeset[
    col sep=comma,
    string type,
    columns={Distance,Photosynthese},
    columns/Distance/.style={
        column name={Distance (cm)},
        dec sep align
    },
    columns/Photosynthese/.style={
        column name={Production O$_2$ (mm³/min)},
        dec sep align,
        precision=1,
        dec sep={,},
        fixed,          % ← Ajouter
        fixed zerofill, % ← Ajouter
    },
    every head row/.style={
        before row=\toprule,
        after row=\midrule
    },
    every last row/.style={
        after row=\bottomrule
    },
    every even row/.style={
        before row={\rowcolor{blue!5}}
    }
]{mesures.csv}
\end{center}
